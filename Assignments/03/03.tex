\documentclass[11pt,a4paper]{article}

\usepackage[utf8]{inputenc}
\usepackage{a4wide}
\usepackage{amsmath,amssymb}
\usepackage{boxproof}
\usepackage{daymonthyear}
\usepackage{float} % for the H placement identifier
\usepackage{graphicx}
\usepackage{tikz}
\usepackage{tabu}
\usepackage{xcolor}

%===== SETTINGS =====%

\def\meta#1{\mbox{$\langle\hbox{#1}\rangle$}}
\def\macrowitharg#1#2{{\tt\string#1\bra\meta{#2}\ket}}

{\escapechar-1 \xdef\bra{\string\{}\xdef\ket{\string\}}}

\def\intro#1{{#1}{\cal I}}
\def\elim#1{{#1}{\cal E}}

\showboxbreadth 999
\showboxdepth 999
\tracingoutput 1

\let\imp\to
\def\elim#1{{{#1}{\cal E}}}
\def\intro#1{{{#1}{\cal I}}}
\def\lt{<}
\def\eqdef{\overset{\mathrm{def}}{=\joinrel=}}
\def\eps{\mathrel{\epsilon}}
\def\biimplies{\leftrightarrow}
\def\flt#1{\mathrel{{#1}^\flat}}
\def\setof#1{{\left\{{#1}\right\}}}
\let\implies\to
\def\KK{{\mathsf K}}
\let\squashmuskip\relax

%===== META DATA =====%

\title
{
{\Large Logic in Computer Science}\\
Assignment 3
}
\author
{
	Casper B. Hansen\\
	University of Copenhagen\\
	Department of Computer Science\\
	{\tt fvx507@alumni.ku.dk}
}
\date{\today}

%===== DOCUMENT =====%

\begin{document}

\maketitle

\section*{2.2.3 \mdseries Which of the following strings are formulas in
predicate logic? Specify a reason for failure for strings which aren’t, draw
parse trees of all strings which are.}
% For exercise 2.2.3a you do not have to draw the parse trees.
\subsection*{(a) \mdseries Let $m$ be a constant, $f$ a function symbol with
one argument and $S$ and $B$ two predicate symbols, each with two arguments:
\\
\begin{tabular}{rl}
	i. & $S(m, x)$ \\
	ii. & $B(m, f(m))$ \\
	iii. & $f(m)$ \\
	iv. & $B(B(m, x), y)$ \\
	v. & $S(B(m), z)$ \\
	vi. & $(B(x, y) \imp (\exists z{\ }S(z,y)))$ \\
	vii. & $(S(x,y) \imp S(y, f(f(x))))$ \\
	viii. & $(B(x) \imp B(B(x))$ \\
\end{tabular}
}
The first {\it i} and second {\it ii} are formulae, as both $S$ and $B$ has
arity 2, and is supplied with two arguments; namely the variable $x$ and the
constant $m$. For the second argument this is just a unary function, which is
valid for reasons given in {\it iii}.

The third {\it iii} is a term, but {\it not} a formula. The fourth {\it iv} is
{\it not} a formula as the outer $B$ is supplied with a predicate, rather than
a term.

The fifth {\it v} is {\it not} a formula, as them arguments supplied does not
match the arity of $B$. The sixth {\it vi} is a formula, as the left-hand side
of the implication is valid for the same reasons as given for {\it i}, and the
existence declaration does not change the fact that the right-hand side is
valid.

The seventh {\it vii} is a formula; the left-hand side is valid for the same
reason as that of {\it i}, the right-hand side will evaluate the inner $f$,
which in turn supplies still just one argument to the outer $f$, and then, for
the exact same reason as the left-hand side, so is the right-hand side valid.
The implication is valid because the both sides of it are valid formulae.

The eighth, and last, {\it viii} is {\it not} a formula, as it violates the
arity of $B$.

\section*{2.2.4 \mdseries Let $\phi$ be $\exists x(P(y,z) \land
(\forall y(\neg Q(y,x) \lor P(y,z))))$, where $P$ and $Q$ are predicate
symbols with two arguments.}
% 4cd you should use the unsafe substitution (Definition 2.7 in H+R, p.105),
% not the safe one from the lectures.
\subsection*{(c) \mdseries Is there a variable in $\phi$ which has free and
bound occurrences?}
Yes. The variable $y$ is free in the scope of $\exists x \phi$, whereas it is
bound under the scope of $\forall y \phi$.

\subsection*{(d) \mdseries Consider the terms $w$ ($w$ is a variable), $f(x)$
and $g(y,z)$, where $f$ and $g$ are function symbols with arity $1$ and $2$,
respectively.
\\
\begin{tabular}{rl}
	i. & Compute $\phi[w/x]$, $\phi[w/y]$, $\phi[f(x)/y]$ and $\phi[g(y,z)/z]$. \\
	ii. & Which of $w$, $f(x)$ and $g(y,z)$ are free for $x$ in $\phi$? \\
	iii. & Which of $w$, $f(x)$ and $g(y,z)$ are free for $y$ in $\phi$?
\end{tabular}
}
For $\phi[w/x]$ we have no free $x$ in our formula, and hence the formula
doesn't change at all.
\begin{align*}
	\phi[w/x] &= \exists x(P(y,z) \land (\forall y(\neg Q(y,x) \lor P(y,z))))
\end{align*}
In $\phi[w/y]$ however, we have one free $y$ on the left-hand side of the upper
conjunction, and so we replace this $y$ with the variable $w$.
\begin{align*}
	\phi[w/y] &= \exists x(P(w,z) \land (\forall y(\neg Q(y,x) \lor P(y,z))))
\end{align*}
We notice a problem with $\phi[f(x)/y]$, as we replace the $y$ by $f(x)$,
since $x$ is bound under the scope. But the assignment explicitly tells us to
use the {\it unsafe} substitution method (by definition 2.7), and so we simply
ignore the problem and perform the substitution.
\begin{align*}
	\phi[f(x)/y] &= \exists x(P(f(x),z) \land (\forall y(\neg Q(y,x) \lor P(y,z))))
\end{align*}
For $\phi[g(y,z)/z]$ both occurences of $z$ are free, and as such, we can
simply replace them. Yet again, there is a problem with $y$ being bound, but
we can ignore this since we are employing the {\it unsafe} substitution method.
\begin{align*}
	\phi[g(y,z)/z] &= \exists x(P(y,g(y,z)) \land (\forall y(\neg Q(y,x) \lor P(y,g(y,z)))))
\end{align*}

No bindings on $w$ exists in $\phi$, and therefore $w$ is free for both $x$
and $y$. For $f(x)$, since there is a binding on $x$ at the root, that is, the
entire scope of the formula, $f(x)$ isn't free for any substitution, including
$x$ and $y$. Since no bindings are present in $\phi$ on $z$ we can disregard
it when considering whether or not $g(y,z)$ is free for some variable, but we
do have to consider the $y$; on the right-hand side of the upper most
conjunction we have a binding of $y$ in its scope, and a leaf $y$, and so
$g(y,z)$ is not free for $y$. The case of $g(y,z)$ being free for $x$ is the
same argument, by replacing $y$ with $x$, as we also have a leaf of $x$ in
that subtree.

% For exercises 2.3 1a 7c 9al formal errors will not be tolerated.
% For this reason we strongly encourage you to use ProofWeb to generate
% and/or verify your proofs.

\newpage
\section*{2.3.1 \mdseries Prove the validity of the following sequents using,
among others, the rules $\intro=$ and $\elim=$. Make sure that you indicate
for each application of $\elim=$ what the rule instances $\phi$, $t_1$ and
$t_2$ are.}
% Formal errors not tolerated!

\subsection*{(a) \mdseries $(y = 0) \land (y = x) \vdash 0 = x$}
\begin{proofbox}
	\lbl{1}	\: (y = 0) \land (y = x)			\=\mbox{premise} \\
	\lbl{2}	\: y = 0							\=\elim\land(\ref{1}) \\
	\lbl{3}	\: y = x							\=\elim\land(\ref{1}) \\
	\lbl{4}	\: x = 0							\=\elim=(1,2) \\
	% not sure if this these next steps are necessary
	\lbl{5}	\: x = x							\=\intro= \\
	\lbl{6}	\: 0 = x							\=\elim=(4,5) \\
\end{proofbox}

\section*{2.3.7 \mdseries The sequents below look a bit tedious, but in
proving their validity you make sure that you really understand how to nest
the proof rules.}
% Formal errors not tolerated!

\subsection*{(c) \mdseries $\exists x{\ }\forall y{\ }P(x,y) \vdash
\forall y{\ }\exists x{\ }P(x,y)$}
\begin{proofbox}
	\lbl{1} \: \exists x{\ }\forall y{\ }P(x,y)		\=\mbox{premise} \\
	\[
y_0	\lbl{2} \: 										\=\mbox{} \\
	\[
x_0	\lbl{3} \: \forall y{\ }P(x_0, y)				\=\mbox{assumption} \\
	\lbl{4} \: P(x_0, y_0)							\=\elim\forall(\ref{3}) \\
	\lbl{5} \: \exists x{\ }P(x,y_0)				\=\intro\exists(\ref{4}) \\
	\]
	\lbl{6} \: \exists x{\ }P(x,y_0)				\=\elim\exists(\ref{1}, 3-5) \\
	\]
	\lbl{7} \: \forall y{\ }\exists x{\ }P(x,y)		\=\intro\forall(2-6) \\
\end{proofbox}

\newpage
\section*{2.3.9 \mdseries Prove the validity of the following sequents in
predicate logic, where $F$, $G$, $P$, and $Q$ have arity $1$, and $S$ has
arity $0$ (a ‘propositional atom’).}
% Formal errors not tolerated!

\subsection*{(a) \mdseries $\exists x{\ }(S \imp Q(x)) \vdash S \imp
\exists x{\ }Q(x)$}
\begin{proofbox}
	\lbl{1}	\: \exists x{\ }(S \imp Q(x))			\=\mbox{premise} \\
	\[
	\lbl{2}	\: S									\=\mbox{assumption} \\
	\[
x_0	\lbl{3}	\: S \imp Q(x_0)						\=\mbox{assumption} \\
	\lbl{4}	\: Q(x_0)								\=\elim\imp(\ref{3}) \\
	\lbl{5}	\: \exists Q(x)							\=\intro\exists(\ref{4}) \\
	\]
	\lbl{6}	\: \exists Q(x)							\=\elim\exists(1, 3-5) \\
	\]
	\lbl{7}	\: S \imp \exists x{\ }Q(x)				\=\intro\imp(2-6) \\
\end{proofbox}

\subsection*{(l) \mdseries $\forall x{\ }P(x) \lor \forall x{\ }Q(x)
\vdash \forall x{\ }(P(x) \lor Q(x))$}
\begin{proofbox}
	\lbl{1}	\: \forall x{\ }P(x) \lor \forall x{\ }Q(x)		\=\mbox{premise} \\
	\[
x_0	\lbl{2} \: \mbox{}										\=\mbox{} \\
	\(
	\lbl{3}	\: \forall P(x)									\=\mbox{assumption} \\
	\lbl{4}	\: P(x_0)										\=\elim\forall(\ref{1}) \\
	\lbl{5}	\: P(x_0) \lor Q(x_0)							\=\intro\lor(\ref{4}) \\
	\*
	\lbl{3}	\: \forall Q(x)									\=\mbox{assumption} \\
	\lbl{4}	\: Q(x_0)										\=\elim\forall(\ref{1}) \\
	\lbl{5}	\: P(x_0) \lor Q(x_0)							\=\intro\lor(\ref{4}) \\
	\)
	\lbl{6} \: P(x_0) \lor Q(x_0)							\=\elim\lor(\ref{1}, 3-5) \\
	\]
	\lbl{7}	\: \forall x{\ }( P(x) \lor Q(x) )				\=\intro\forall(2-6) \\
\end{proofbox}

\section*{2.4.5 \mdseries Let $\phi$ be the sentence $\forall x{\ }\forall
y{\ }\exists z{\ }(R(x, y) \imp R(y, z))$, where $R$ is a predicate symbol
of two arguments}
% For exercise 2.4: 5 you need not go into the same excruciating level of
% detail as for the Erasmus Montanus exercise.

% all
\subsection*{(a) \mdseries Let $A \eqdef \{a,b,c,d\}$ and $R^{\cal M} \eqdef
\{(b,c),(b,b),(b,a)\}$. Do we have $\cal M \models \phi$? Justify your answer,
whatever it is.}
No, for the simple reason that choosing $R(b,c)$ leaves us with no tubles in
$R^n$ for the $y$ in the implication --- that is, there exists no such tuble
which has $c$ as its first argument.

\subsection*{(b) \mdseries Let $A \eqdef \{a,b,c\}$ and $R^{\cal M'} \eqdef
\{(b,c),(a,b),(c,b)\}$. Do we have $\cal M' \models \phi$? Justify your answer,
whatever it is.}
The converse is true for this one, as no matter which tuble we choose, the
implication still holds, as there are tuples in the set that satisfies the
right-hand side.

\section*{2.4.8 \mdseries Show the semantic entailment $\forall x{\ }P (x)
\lor \forall x{\ }Q(x) \models \forall x{\ }(P (x) \lor Q(x))$}
% a detail level matching that of similar proof from the slides from Monday 16
% is sufficient.
For a model $\cal M$ satisfying $\forall x{\ }P (x) \lor \forall x{\ }Q(x)$,
we must have that ${\cal M} \models \forall x{\ }P (x)$ and ${\cal M} \models
\forall x{\ }Q(x)$. So all elements of $A$ must be a subset of the union of
those --- that is, $A \subseteq P^n \cup Q^n$, implying that ${\cal M} \models
\forall x{\ }(P(x) \lor Q(x))$ holds.

\section*{2.4.10 \mdseries Is $\forall x{\ }(P (x) \lor Q(x)) \models \forall
x{\ }P(x) \lor \forall x{\ }Q(x)$ a semantic entailment? Justify your answer.}
% a detail level matching that of similar proof from the slides from Monday 16
% is sufficient.
No, because we cannot argue it reasonably. We would need a copy-paste argument
such as the one given above, but this isn't possible because of its structure.

\newpage
\section*{Additional exercise \mdseries Structural induction}
We define the height $h(\phi)$ of a propositional formula $\phi$ as follows.
\begin{align*}
	h(\phi) =
	\begin{cases}
		0
		& \mbox{if $\phi = \bot$ or $\phi = p$} \\
		%
		1 + h(\phi)
		& \mbox{if $\phi = \neg \phi$} \\
		%
		1 + \text{max}\{h(\phi_1), h(\phi_2)\}
		& \mbox{if $\phi = \Gamma$}
	\end{cases}
\end{align*}
where $\Gamma$ is any one of the binary logical operations $\land$, $\lor$ or
$\imp$ on $\phi_1$ and $\phi_2$.

\subsection*{Use structural induction to show that any propositional logic
formula $\phi$ of height $h$ contains at most $2^h$ propositional atoms. (You
may argue that some of the cases are similar.)}
Firstly, let us recognize that propositional atoms for which the first case
applies doesn't increase the height of the parse tree, as such the proof must
rely on the structural composition of its logical connectives (ie. $\neg$,
$\land$, etc.).

In order to show that the height of any formula $\phi$ contains at most $2^h$
propositional atoms we must define such an instance in which propositional
atom occurrences is maximized, while minimizing the height of the tree. This
occurs only when the parse tree mimics a complete binary tree. That is, all
non-leaf nodes have two children and the last level is filled by leaves.

\begin{description}
	\item {\bf Inductive hypothesis} Let $L(x)$ and $R(x)$ denote the number
	of propositional atoms for the left and right subtrees, respectively, of
	the parse tree rooted at $x$. Then for a fully balanced complete binary
	(and hence only uses binary logical connectives) parse tree $L(x) = R(x)
	= 2^{h(x) - 1}$, and thereby $L(x) + R(x) = 2^{h(x)}$.
	
	\item {\bf Base case} For the base case, we have that when $\phi$ is a
	propositional atom the height is left unchanged at $0$, since no
	subtrees exists, and hence trivially satisfies the hypothesis.
	
	\item {\bf Inductive step} Consider a full binary parse tree. For this
	to be the case it must consist only by binary logical connectives by
	definition --- that is, any and all logical connective in $\Gamma$ will
	do. Let $x$ be the root node in any such parse, and let $y$ be that same
	node, but in a copy of the former parse tree in which we replace each
	propositional atom by a binary logical connective, and further let its
	two arguments be propositional atoms.

	We then have that $L(x) = 2^{h(x) - 1}$ and $L(y) = 2^{h(y) - 1}$.

	Since we added a level to the height of the tree rooted at $x$, we
	can express it as follows
	\begin{align*}
		L(x) &= 2 \cdot 2^{h(x) - 1} = 2^{h(x)} = 2^{h(y) - 1} = L(y)
	\end{align*}
	We see that it is exactly $L(y)$ from which the inductive hypothesis
	follows.
\end{description}

\subsection*{Use structural induction to show that if $\phi$ contains the
connective $\neg$ anywhere, then $\phi$ contains strictly less than $2^h$
propositional atoms. (Hint: use the result from 1.)}
Suppose there is a complete binary parse tree, where the number of leaves
equal $2^h$, where one of the logical connectives is a $\neg$. Then replacing
it by a binary connective would make it possible to have $2^h + 1$ atoms for
a formula $\phi$, which is a contradiction to the former proof.

\end{document}