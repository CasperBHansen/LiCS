\section{CTL (computation tree logic)}
This part of the exam is about CTL, computation tree logic. For convenience
we reproduce a 0-based CTL semantics here (as opposed to Huth+Ryan who use a
1-based notation.) Let ${\cal M} = (S,\imp,L)$ be a model. Recall that
$P_{\cal M}(s)$ is the set of paths in model ${\cal M}$ starting with state
$s$, and that $\sigma[i]$ is the $i$th element (state) of path $\sigma$
(0-indexed in the semantics below, 1-indexed in Huth+Ryan).

{\bf 0-based CTL semantics:}
\begin{align*}
	{\cal M}, s &\models \top \\
	{\cal M}, s &\not\models \bot \\
	{\cal M}, s &\models p \qquad\Leftrightarrow\qquad p \in L(s) \\
	{\cal M}, s &\models \neg\phi \qquad\Leftrightarrow\qquad
		{\cal M}, s \not\models \phi \\
	{\cal M}, s &\models \phi_1 \land \phi_2 \qquad\Leftrightarrow\qquad
		{\cal M}, s \models \phi_1 \land {\cal M}, s \models \phi_2 \\
	&\vdots
\end{align*}
You may use either the 0-based or 1-based semantics to answer this part of the
exam, but be sure to clearly mark which one you use in your answer.

\subsection*{Question 5.1 \mdseries Provide a single model ${\cal M}_1 =
(S_1, \imp_1, L_1)$ and a state $s_1 \in S_1$ in this model such that
${\cal M}_1, s_1 \models \text{\tt EXAG} p$ and ${\cal M}_1, s_1 \not\models
\text{\tt AXEF} p$. That is, in your model ${\cal M}_1, s_1$ should satisfy
the CTL formula $\text{\tt EXAG} p$ but $s_1$ must not satisfy
$\text{\tt AXEF} p$. \newline\indent
Show correctness of your answer by providing a description of ${\cal M}_1$
with the CTL formulas and running this in NuSMV. Your answer must contain both
the NuSMV description and the result of running you description.}
...

\subsection*{Question 5.2}

\subsection*{Question 5.3 \mdseries The Boolean function $g$ of 4 arguments
is defined by \[g(p,q,p',q') = p \cdot \overline{q'} + \overline{p} \cdot q'\]
where we follow the notation from representation of transistion functions from
symbolic model checking, such that $x'$ (where $x$ is an atom) denotes the
value of the next state.}
\subsubsection*{(a) \mdseries Construct a model ${\cal M}_g =
(S_g, \imp_g, L_g)$ such that $g$ represents the transition relation $\imp_g$
of the model. You may assume that the set of atomic propositions is ${p, q}$,
thus your model should have 4 states.}
\subsubsection*{(b) \mdseries Implement the model ${\cal M}_g$ in NuSMV with
all states as starting state. You may either use the 'stateful' or 'stateless'
approach (see first lecture on NuSMV).\newline\indent
[Hint: If you choose the stateless approach, then consider what the value of
$p$ tells you about the next value of $q$. This can give you a one-line
implementations of the next functions. You are welcome to use the template.}
\begin{figure}[H]
	\lstinputlisting{M_g.smv}
	\label{code:M_g}
	\caption{NuSMV code of the ${\cal M}_g$ model}
\end{figure}
\subsubsection*{(c) \mdseries ...}
...
