\section{The SAT solver}
This question concerns the SAT solver {\it satisfiability algorithm} for
propositional logic formulas. Here, we shall consider propositional formulas
to be generated by the following grammar
\[ \phi::={\ }p{\ }|{\ }\neg\phi{\ }|{\ }\phi \land \phi{\ }|{\ }\phi \uparrow \phi \]
i.e., the well-known $\neg$ and $\land$ (which form an adequate subset), and a
new binary connective $\uparrow$. This new connective (called Scheffer’s
stroke, or nand) has the following truth table.
\begin{center}
\begin{tabular}{c|c||c}
	$\phi$ & $\psi$ & $\phi \uparrow \psi$ \\
	{\tt F} & {\tt F} & {\tt T} \\
	{\tt F} & {\tt T} & {\tt T} \\
	{\tt T} & {\tt F} & {\tt T} \\
	{\tt T} & {\tt T} & {\tt F}
\end{tabular}
\end{center}
Formulas generated by the grammar above can be translated into formulas
without $\uparrow$ by the transformation
\begin{align*}
	T(\phi \uparrow \psi) &= \neg(T(\phi) \land T(\psi)) \\
	T(\neg\phi) &= \neg(T(\phi)) \\
	T(\phi \land \psi) &= T(\phi) \land T(\psi) \\
	T(p) &= p \\
\end{align*}

\subsection*{Question 2.1}
\subsubsection*{(a) \mdseries Show, using structural induction, that
$T(\cdot)$ is correct, i.e., that $T(\phi)$ and $\phi$ are equivalent
formulas $(\phi \equiv T(\phi))$, and that $T(\phi)$ does not contain
$\uparrow$ as a connective.}
Using the grammar, as defined on $\phi$, we write up the structural induction.
\begin{enumerate}
	\item $T(p) \models p$ and $T(\neg p) \models \neg p$
	\item $T(\neg \phi) \models \neg (T(\phi))$
	\item $T(\phi_1 \land \phi_1) \models T(\phi_1) \land T(\phi_2)$
	iff. $\models \phi_1$ and $\models \phi_2$
	\item $T(\phi_1 \uparrow \phi_1) \models \neg(\phi_1 \land \phi_2)$
	iff. $\models \neg \phi_1$ and $\models \neg \phi_2$
\end{enumerate}
By the structural induction above, we see that $\phi \equiv T(\phi)$. Also,
no conditions need be satisfied for $T(\cdot)$ to hold in which $\uparrow$
occurs. Hence $T(\phi)$ does not contain the $\uparrow$ as a connective.

\subsubsection*{(b) \mdseries ...}

\subsection*{Question 2.2}
\subsubsection*{(a) \mdseries Create {\it forcing rules} for $\uparrow$.}
\usetikzlibrary{arrows}
\begin{figure}[H]
	\center
	\begin{multicols}{3}
	\begin{tikzpicture}
	[
	align=center,
	node/.style={},
	]
		% connective
		\node[node] (cv) at 		( 0.0,  0.0 ) {\tt F};
		\node[node] (cs) at 		( 0.0, -0.5 ) {$\uparrow$};
		
		% left-hand side
		\node[node] (lv) at 		(-1.0, -1.0 ) {\tt T};
		\node[node] (ls) at 		(-1.0, -1.5 ) {$\circ$};
		
		% right-hand side
		\node[node] (rv) at 		( 1.0, -1.0 ) {\tt T};
		\node[node] (rs) at 		( 1.0, -1.5 ) {$\circ$};
		
		\foreach \from/\to in {cs/ls,cs/rs} \draw[solid] (\from) -- (\to);
		% \foreach \from/\to in {cv/lv,cv/rv} \draw[double] (\from) -- (\to);
		
	\end{tikzpicture}\\
	When the connective node is {\tt F}, it forces both subnodes to become
	{\tt T}.
	
	\vfill
	\columnbreak
	
	\begin{tikzpicture}
	[
	align=center,
	node/.style={},
	]
		% connective
		\node[node] (cv) at 		( 0.0,  0.0 ) {\tt T};
		\node[node] (cs) at 		( 0.0, -0.5 ) {$\uparrow$};
		
		% left-hand side
		\node[node] (lv) at 		(-1.0, -1.0 ) {\tt F};
		\node[node] (ls) at 		(-1.0, -1.5 ) {$\circ$};
		
		% right-hand side
	%	\node[node] (rv) at 		( 1.0, -1.0 ) {\tt F};
		\node[node] (rs) at 		( 1.0, -1.5 ) {$\circ$};
		
		\foreach \from/\to in {cs/ls,cs/rs} \draw[solid] (\from) -- (\to);
	%	\foreach \from/\to in {cv/lv,cv/rv} \draw[double] (\from) -- (\to);
		
	\end{tikzpicture}\\
	If either of the connectives subnodes are {\tt F}, it then forces the
	node to become {\tt T}.
	
	\vfill
	\columnbreak
	
	\begin{tikzpicture}
	[
	align=center,
	node/.style={},
	]
		% connective
		\node[node] (cv) at 		( 0.0,  0.0 ) {\tt T};
		\node[node] (cs) at 		( 0.0, -0.5 ) {$\uparrow$};
		
		% left-hand side
		\node[node] (lv) at 		(-1.0, -1.0 ) {\tt T};
		\node[node] (ls) at 		(-1.0, -1.5 ) {$\circ$};
		
		% right-hand side
		\node[node] (rv) at 		( 1.0, -1.0 ) {\tt F};
		\node[node] (rs) at 		( 1.0, -1.5 ) {$\circ$};
		
		\foreach \from/\to in {cs/ls,cs/rs} \draw[solid] (\from) -- (\to);
	%	\foreach \from/\to in {cv/lv,cv/rv} \draw[double] (\from) -- (\to);
		
	\end{tikzpicture}\\
	If we know the connective node to be {\tt T}, and one of its subnodes is
	also {\tt T}, then it forces the other to be {\tt F}.
	
	\end{multicols}
	
	\begin{multicols}{3}
		
	\begin{tikzpicture}
	[
	align=center,
	node/.style={},
	]
		% connective
		\node[node] (cv) at 		( 0.0,  0.0 ) {\tt F};
		\node[node] (cs) at 		( 0.0, -0.5 ) {$\uparrow$};
		
		% left-hand side
		\node[node] (lv) at 		(-1.0, -1.0 ) {\tt T};
		\node[node] (ls) at 		(-1.0, -1.5 ) {$\circ$};
		
		% right-hand side
		\node[node] (rv) at 		( 1.0, -1.0 ) {\tt T};
		\node[node] (rs) at 		( 1.0, -1.5 ) {$\circ$};
		
		\foreach \from/\to in {cs/ls,cs/rs} \draw[solid] (\from) -- (\to);
		% \foreach \from/\to in {cv/lv,cv/rv} \draw[double] (\from) -- (\to);
		
	\end{tikzpicture}\\
	The opposite is true, that whenever its subnodes are both {\tt T}, then it
	follows that the connective node must be {\tt F}.
	
	\vfill
	\columnbreak
		
	\begin{tikzpicture}
	[
	align=center,
	node/.style={},
	]
		% connective
		\node[node] (cv) at 		( 0.0,  0.0 ) {\tt T};
		\node[node] (cs) at 		( 0.0, -0.5 ) {$\uparrow$};
		
		% left-hand side
	%	\node[node] (lv) at 		(-1.0, -1.0 ) {\tt F};
		\node[node] (ls) at 		(-1.0, -1.5 ) {$\circ$};
		
		% right-hand side
		\node[node] (rv) at 		( 1.0, -1.0 ) {\tt F};
		\node[node] (rs) at 		( 1.0, -1.5 ) {$\circ$};
		
		\foreach \from/\to in {cs/ls,cs/rs} \draw[solid] (\from) -- (\to);
	%	\foreach \from/\to in {cv/lv,cv/rv} \draw[double] (\from) -- (\to);
		
	\end{tikzpicture}\\
	If either of the connectives subnodes are {\tt F}, it then forces the
	node to become {\tt T}.
	
	\vfill
	\columnbreak
	
	\begin{tikzpicture}
	[
	align=center,
	node/.style={},
	]
		% connective
		\node[node] (cv) at 		( 0.0,  0.0 ) {\tt T};
		\node[node] (cs) at 		( 0.0, -0.5 ) {$\uparrow$};
		
		% left-hand side
		\node[node] (lv) at 		(-1.0, -1.0 ) {\tt F};
		\node[node] (ls) at 		(-1.0, -1.5 ) {$\circ$};
		
		% right-hand side
		\node[node] (rv) at 		( 1.0, -1.0 ) {\tt T};
		\node[node] (rs) at 		( 1.0, -1.5 ) {$\circ$};
		
		\foreach \from/\to in {cs/ls,cs/rs} \draw[solid] (\from) -- (\to);
	%	\foreach \from/\to in {cv/lv,cv/rv} \draw[double] (\from) -- (\to);
		
	\end{tikzpicture}\\
	If we know the connective node to be {\tt T}, and one of its subnodes is
	also {\tt T}, then it forces the other to be {\tt F}.
	
	\end{multicols}
	\label{fig:uparrow-forced-rules}
	\caption{Forced rules for the $\uparrow$ connective}
\end{figure}

\subsubsection*{(b) \mdseries ...}
\subsubsection*{(c) \mdseries ...}

\subsection*{Question 2.3 \mdseries The {\it cubic} SAT solver algorithm
allows you to guess nodes in {\it any} order. Argue (informally) for why the
order does not affect the result of the algorithm.}
I'm inclined to argue by algorithmic theory; the reason is that the algorithm
relies on the principles of {\it dynamic programming}, which produce their
results by trying all possible solutions.

