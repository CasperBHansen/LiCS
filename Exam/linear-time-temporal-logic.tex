\section{LTL (linear-time temporal logic)}

\[
	\pi \models \phi \text{\tt D} \psi
	\quad \text{iff.} \quad
	\forall j \geq 0.(\pi^j \models \phi \imp
	[\exists k \geq 0.\exists l > 0.(\pi^{j+k} \models \psi)
	\land (\pi^{j+k+l} \models \psi)])
\]

\subsection*{Question 4.1}
We define the LTL formula of the {\tt D} connective as such
\begin{align*}
	\text{\tt G}( \phi \imp \text{\tt F}\psi \land \text{\tt F}(\text{\tt X}\psi) )
\end{align*}
which states that in all future states from which $\phi$ holds, it follows
that there exists a future state such that $\psi$ holds and there exists a
future state for which the next $\psi$ holds.

\subsection*{Question 4.2 \mdseries Using the {\it semantics} of {\tt D}
defined above, prove that \[(\phi \land \psi)\text{\tt D}\eta \equiv
(\phi\text{\tt D}\eta) \land (\psi\text{\tt D}\eta)\]
Arguments not based on the semantics will not receive points}

By the semantics defined for {\tt D} we have, for $(\phi \land \psi)
\text{\tt D} \eta$, that $\eta$ is satisfied twice for some $k \geq 0$ and
some $l > 0$. Further we have that $\phi$ is satisfied for some $j \geq 0$,
since $\phi \land \psi$ is satisfied, reasoned by $\phi \land \psi \imp \phi$.
The above reasoning is the equivalent of stating that $\phi \text{\tt D}
\eta$. The argument that is also satisfied $\psi \text{\tt D} \eta$ is
symmetric. Since both hold, we can conclude that $(\phi \land \psi)
\text{\tt D}\eta \equiv (\phi\text{\tt D}\eta) \land (\psi\text{\tt D}\eta)$.
