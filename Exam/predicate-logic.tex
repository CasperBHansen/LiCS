\section{Predicate logic}
This part of the assignment concerns predicate logic. Remember that for
predicate logic (with equality) we extend natural deduction with rules for
$\forall$, $\exists$, $=$ and that arguing about models (the semantics of
predicate) logic is very different from the semantics of propositional logic
(e.g. Example 2.21 in Huth+Ryan.).

\subsection*{Question 3.1}
In the following let
\begin{itemize}
	\item $\phi$ denote $\exists v \forall w L(v,w)$
	\item $\psi$ denote $\forall w \exists v L(v,w)$
	\item $\eta$ denote $\forall x \forall y \exists z (L(z,x) \land L(z,y))$
\end{itemize}

\subsubsection*{(a) \mdseries Using natural deduction for predicate logic,
prove the sequent $\phi \vdash \eta$, ie., that \[\exists v \forall w L(v,w)
\vdash \forall x \forall y \exists z (L(z,x) \land L(z,y))\]}
\begin{proofbox}
	\lbl{1} \: \exists v \forall w L(v, w) 				\=\mbox{premise} \\
\[
x_0 \lbl{2} \: \=\mbox{\ }
\[
y_0 \lbl{3} \: \=\mbox{\ }
\[
z_0 \lbl{4} \: \forall w L(z_0, w) 				\=\mbox{assumption} \\
	\lbl{5} \: L(z_0, x_0) 						\=\elim\forall_w(\ref{4}) \\
	\lbl{6} \: L(z_0, y_0) 						\=\elim\forall_w(\ref{4}) \\
	\lbl{7} \: L(z_0, x_0) \land L(z_0, y_0) 	\=\intro\land(\ref{5},\ref{6}) \\
	\lbl{8} \: \exists z (L(z,x_0) \land L(z,y_0)) 	\=\intro\exists_z(7) \\
\]
	\lbl{9} \: \exists z (L(z,x_0) \land L(z,y_0)) 		\=\elim\exists(1, 4-8) \\
\]
	\lbl{10} \: \forall y \exists z (L(z,x_0) \land L(z,y)) 	\=\intro\forall_y(3-9) \\
\]
	\lbl{11} \: \forall x \forall y \exists z (L(z,x) \land L(z,y)) 	\=\intro\forall_x(2-10) \\
\end{proofbox}

\subsubsection*{(b) \mdseries Show that $\psi \not\vdash \eta$, i.e., that
$\forall w \exists v L(v,w) \not\vdash \forall x \forall y \exists z (L(z,x)
\land L(z,y))$}
...

\subsubsection*{(c) \mdseries Assuming $\psi \vdash \eta$ and $\psi \not\vdash
\eta$ (the results from (a) and (b) above) argue that $\not\vdash \psi \imp
\phi$, i.e., that $\not\vdash \forall w \exists v L(v,w) \imp \exists v
\forall w L(v,w)$}
...

\subsection*{Question 3.2}
In this subquestion we consider the predicate logic formula
\[\exists x (P(x) \imp \forall y P(y))\]
This is a valid predicate logic formula which we shall call "The Peter
Principle." To understand the formula a little better, consider a natural
language interpretation where $P(x)$ means “person x is called Peter.” Under
this reading the above formula expresses that “there is a person such that if
that person is called Peter then everybody is called Peter.” Pay close
attention to the parentheses: the formula is not $\exists x P(x) \imp
\forall y P(y)$, which is very much invalid.

Your task is now to show formally that $\exists x (P(x) \imp \forall y P(y))$
is valid in two different ways.

\subsubsection*{(a) \mdseries Using the semantics of predicate logic, show
that $\models \exists x (P(x) \imp \forall y P(y))$ holds. That is, given an
arbitrary model ${\cal M} = (A, \cdot, \{P^{\cal M}\})$ you must show that
${\cal M} \models \exists x (P(x) \imp \forall y P(y))$.\newline
[Hint: Start by considering that either $P^{\cal M} = A$ or
$P^{\cal M} \not= A$.]}

\subsubsection*{(b) \mdseries Using natural deduction for predicate logic,
prove the sequent $\vdash \exists x (P(x) \imp \forall y P(y))$. In this proof
you may, exceptionally, use the equivalence $\neg \forall y P(y) \dashv\vdash
\exists y \neg P(y)$ (see p. 119 of Huth+Ryan for the proof of this
equivalence.)}
\begin{proofbox}
\[
x_0 \lbl{1} \: P(x_0) 							\=\mbox{assumption} \\
\]
	\lbl{.} \: \exists x (P(x) \imp \forall y P(y)) \=\exists\elim{...}
\end{proofbox}
