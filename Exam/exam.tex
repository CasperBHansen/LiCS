\documentclass[10pt,a4paper]{article}

\usepackage[utf8]{inputenc}
\usepackage{a4wide}
\usepackage{amsmath,amssymb}
\usepackage{boxproof}
\usepackage{daymonthyear}
\usepackage{enumerate}
\usepackage{float} % for the H placement identifier
\usepackage{stmaryrd}

%===== SETTINGS =====%

\def\meta#1{\mbox{$\langle\hbox{#1}\rangle$}}
\def\macrowitharg#1#2{{\tt\string#1\bra\meta{#2}\ket}}

{\escapechar-1 \xdef\bra{\string\{}\xdef\ket{\string\}}}

\def\intro#1{{#1}{\cal I}}
\def\elim#1{{#1}{\cal E}}

\showboxbreadth 999
\showboxdepth 999
\tracingoutput 1

\let\imp\to
\def\elim#1{{{#1}{\cal E}}}
\def\intro#1{{{#1}{\cal I}}}
\def\lt{<}
\def\eqdef{\overset{\mathrm{def}}{=\joinrel=}}
\def\eps{\mathrel{\epsilon}}
\def\biimplies{\leftrightarrow}
\def\flt#1{\mathrel{{#1}^\flat}}
\def\setof#1{{\left\{{#1}\right\}}}
\let\implies\to
\def\KK{{\mathsf K}}
\let\squashmuskip\relax

%===== META DATA =====%

\title
{
	{\Large Logic in Computer Science}\\
	Exam
}
\author
{
	Casper B. Hansen\\
	University of Copenhagen\\
	Department of Computer Science\\
	{\tt fvx507@alumni.ku.dk}
}
\date{\today}

%===== DOCUMENT =====%

\begin{document}

\maketitle

\section{Propositional logic}
\subsection*{Question 1.1}
This subquestion concerns the use of {\it proof theoretic} argument in
propositional logic, i.e. arguments relating to natural deduction. Note that
you may {\it only} use the rules of page 27 in Huth+Ryan. You are not allowed
to shortcut your formal proofs with equivalences.
\subsubsection*{(a) \mdseries Use natural deduction for propositional logic
(i.e. the deduction system on page 27 of Huth+Ryan) to show that the following
sequents are valid.}

\begin{enumerate}[i]
	\item
	{
	$\vdash \neg\neg p \land q \imp p \lor r$
	\begin{proofbox}
	\[
		\lbl{1} \: \neg\neg p \land q 			\=\mbox{assumption} \\
		\lbl{2} \: \neg\neg p 					\=\elim\land_1(\ref{1}) \\
		\lbl{3} \: q 							\=\elim\land_2(\ref{1}) \\
		\lbl{4} \: p 							\=\elim{\neg\neg}(\ref{2}) \\
	\(
		\lbl{.} \: p 							\=\mbox{assumption} \\
		\lbl{.} \: p \lor r						\=\intro\lor_1(\ref{?}) \\
	\*
		\lbl{.} \: r							\=\mbox{assumption} \\
		\lbl{.} \: p \lor r						\=\intro\lor_2(\ref{?}) \\
	\)
		\lbl{.} \: p \lor r 					\=\mbox{...} \\
	\]
		\lbl{.} \: (\neg\neg p \land q) \imp p \lor r 	\=\mbox{conclusion} \\
	\end{proofbox}
	}
	\item
	{
	$\neg q \imp \neg p \vdash p \imp q$
	\begin{proofbox}
		\lbl{1} \: \neg q \imp \neg p 			\=\mbox{premise} \\
	\[
		\lbl{2} \: p 							\=\mbox{assumption} \\
	\[
		\lbl{3} \: \neg q 						\=\mbox{assumption} \\
		\lbl{4} \: \neg p 					\=\elim\imp(\ref{1}, \ref{3}) \\
		\lbl{5} \: \bot 					\=\elim\neg(\ref{2}, \ref{4}) \\
	\]
		\lbl{6} \: \neg\neg q 					\=\intro\neg(3-5) \\
		\lbl{7} \: q 							\=\elim{\neg\neg}(\ref{6}) \\
	\]
		\lbl{8} \: p \imp q 					\=\intro\imp(2, 7) \\
	\end{proofbox}
	}
\end{enumerate}

\subsubsection*{(b) \mdseries A deduction rule (without boxes) of form
\[\frac{\phi_1\quad\phi_2\quad\dots\quad\phi_n}{\psi}\text{\scriptsize R}\] is
called {\it derivable} if the conclusion $\psi$ can be derived from the
premises $\phi_1,\dots,\phi_n$ without using rule R. In other words, if you
can provide a proof schema for $\phi_1,\phi_2,\dots,\phi_n\vdash\psi$, using
just the rules of natural deduction for propositional logic, then rule R is
derivable.
\newline\indent
As an example, the rule \[\frac{\phi\imp\neg\phi}{\neg\phi}\text{\scriptsize NREL}\] is
derivable by the proof schema
\begin{proofbox}
	\lbl{1}	\: \phi \imp \neg\phi 				\=\mbox{premise} \\
\[
	\lbl{2}	\: \phi 							\=\mbox{assumption} \\
	\lbl{3}	\: \neg\phi 					\=\elim\imp(\ref{1}, \ref{2}) \\
	\lbl{4}	\: \bot 						\=\elim\neg(\ref{2}, \ref{3}) \\
\]
	\lbl{5}	\: \neg\phi 						\=\intro\neg(2, 4) \\
\end{proofbox}
Show that the following rules are derivable.}

\begin{enumerate}[i]
	\item
	{
	$\frac{\phi}{\psi \imp \phi}\text{\scriptsize WEAK}$
	\begin{proofbox}
		\lbl{1} \: \phi 					\=\mbox{premise} \\
	\[
		\lbl{2} \: \psi 					\=\mbox{assumption} \\
		\lbl{3} \: \phi 					\=\mbox{copy(\ref{1})} \\
	\]
		\lbl{4} \: \psi \imp \phi 			\=\intro\imp(2-3) \\
	\end{proofbox}
	}
	\item
	{
	$\frac{\phi \lor \psi \quad \neg\psi}{\phi}\text{\scriptsize EXT}$
	\begin{proofbox}
		\lbl{1} \: \phi \lor \psi 				\=\mbox{premise} \\
		\lbl{2} \: \neg\psi 					\=\mbox{premise} \\
	\(
		\lbl{3} \: \phi 						\=\mbox{assumption} \\
	\*
		\lbl{3} \: \psi 						\=\mbox{assumption} \\
		\lbl{4} \: \bot 						\=\elim\neg(\ref{2},\ref{3}) \\
		\lbl{5} \: \phi 						\=\elim\bot(\ref{4}) \\
	\)
		\lbl{6} \: \phi 						\=\elim\lor(3-5) \\
	\end{proofbox}
	}
\end{enumerate}

\section{The SAT solver}
\subsection*{Question 2.1}
\subsubsection*{(a) \mdseries ...}
\subsubsection*{(b) \mdseries ...}

\subsection*{Question 2.2}
\subsubsection*{(a) \mdseries ...}
\subsubsection*{(b) \mdseries ...}
\subsubsection*{(c) \mdseries ...}

\subsection*{Question 2.3}

\section{Predicate logic}
\subsection*{Question 3.1}
\subsubsection*{(a) \mdseries ...}
\subsubsection*{(b) \mdseries ...}
\subsubsection*{(c) \mdseries ...}

\subsection*{Question 3.2}
\subsubsection*{(a) \mdseries ...}
\subsubsection*{(b) \mdseries ...}

\section{LTL (linear-time temporal logic)}
\subsection*{Question 4.1}
\subsection*{Question 4.2}
\subsection*{Question 4.3}
\subsubsection*{(a) \mdseries ...}
\subsubsection*{(b) \mdseries ...}

\section{CTL (computation tree logic)}
\subsection*{Question 5.1}
\subsection*{Question 5.2}
\subsection*{Question 5.3}
\subsubsection*{(a) \mdseries ...}
\subsubsection*{(b) \mdseries ...}
\subsubsection*{(c) \mdseries ...}

\section{Representation of Boolean formulas}
\subsection*{Question 6.1}
\subsection*{Question 6.2}
\subsection*{Question 6.3}
\subsection*{Question 6.4}

\end{document}
