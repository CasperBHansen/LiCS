\section{Propositional logic}
\subsection*{Question 1.1}
This subquestion concerns the use of {\it proof theoretic} argument in
propositional logic, i.e. arguments relating to natural deduction. Note that
you may {\it only} use the rules of page 27 in Huth+Ryan. You are not allowed
to shortcut your formal proofs with equivalences.
\subsubsection*{(a) \mdseries Use natural deduction for propositional logic
(i.e. the deduction system on page 27 of Huth+Ryan) to show that the following
sequents are valid.}

\begin{enumerate}[i]
	\item
	{
	$\vdash \neg\neg p \land q \imp p \lor r$
	\begin{proofbox}
	\[
		\lbl{1} \: \neg\neg p \land q 			\=\mbox{assumption} \\
		\lbl{2} \: \neg\neg p 					\=\elim\land_1(\ref{1}) \\
		\lbl{3} \: q 							\=\elim\land_2(\ref{1}) \\
		\lbl{4} \: p 							\=\elim{\neg\neg}(\ref{2}) \\
	\(
		\lbl{5} \: p 							\=\mbox{assumption} \\
		\lbl{6} \: p \lor r						\=\intro\lor_1(\ref{5}) \\
	\*
		\lbl{5} \: r							\=\mbox{assumption} \\
		\lbl{6} \: p \lor r						\=\intro\lor_2(\ref{5}) \\
	\)
		\lbl{7} \: p \lor r 					\=\elim\lor(1, 5-6, 5-6) \\
	\]
		\lbl{8} \: (\neg\neg p \land q) \imp p \lor r 	\=\intro\imp(1-7) \\
	\end{proofbox}
	}
	\item
	{
	$\neg q \imp \neg p \vdash p \imp q$
	\begin{proofbox}
		\lbl{1} \: \neg q \imp \neg p 			\=\mbox{premise} \\
	\[
		\lbl{2} \: p 							\=\mbox{assumption} \\
	\[
		\lbl{3} \: \neg q 						\=\mbox{assumption} \\
		\lbl{4} \: \neg p 					\=\elim\imp(\ref{1}, \ref{3}) \\
		\lbl{5} \: \bot 					\=\elim\neg(\ref{2}, \ref{4}) \\
	\]
		\lbl{6} \: \neg\neg q 					\=\intro\neg(3-5) \\
		\lbl{7} \: q 							\=\elim{\neg\neg}(\ref{6}) \\
	\]
		\lbl{8} \: p \imp q 					\=\intro\imp(2, 7) \\
	\end{proofbox}
	}
\end{enumerate}

\newpage
\subsubsection*{(b) \mdseries A deduction rule (without boxes) of form
\[\frac{\phi_1\quad\phi_2\quad\dots\quad\phi_n}{\psi}\text{\scriptsize R}\] is
called {\it derivable} if the conclusion $\psi$ can be derived from the
premises $\phi_1,\dots,\phi_n$ without using rule R. In other words, if you
can provide a proof schema for $\phi_1,\phi_2,\dots,\phi_n\vdash\psi$, using
just the rules of natural deduction for propositional logic, then rule R is
derivable.
\newline\indent
As an example, the rule \[\frac{\phi\imp\neg\phi}{\neg\phi}\text{\scriptsize NREL}\] is
derivable by the proof schema
\begin{proofbox}
	\lbl{1}	\: \phi \imp \neg\phi 				\=\mbox{premise} \\
\[
	\lbl{2}	\: \phi 							\=\mbox{assumption} \\
	\lbl{3}	\: \neg\phi 					\=\elim\imp(\ref{1}, \ref{2}) \\
	\lbl{4}	\: \bot 						\=\elim\neg(\ref{2}, \ref{3}) \\
\]
	\lbl{5}	\: \neg\phi 						\=\intro\neg(2, 4) \\
\end{proofbox}
Show that the following rules are derivable.}

\begin{enumerate}[i]
	\item
	{
	$\frac{\phi}{\psi \imp \phi}\text{\scriptsize WEAK}$
	\begin{proofbox}
		\lbl{1} \: \phi 					\=\mbox{premise} \\
	\[
		\lbl{2} \: \psi 					\=\mbox{assumption} \\
		\lbl{3} \: \phi 					\=\mbox{copy(\ref{1})} \\
	\]
		\lbl{4} \: \psi \imp \phi 			\=\intro\imp(2-3) \\
	\end{proofbox}
	}
	\item
	{
	$\frac{\phi \lor \psi \quad \neg\psi}{\phi}\text{\scriptsize EXT}$
	\begin{proofbox}
		\lbl{1} \: \phi \lor \psi 				\=\mbox{premise} \\
		\lbl{2} \: \neg\psi 					\=\mbox{premise} \\
	\(
		\lbl{3} \: \phi 						\=\mbox{assumption} \\
	\*
		\lbl{3} \: \psi 						\=\mbox{assumption} \\
		\lbl{4} \: \bot 						\=\elim\neg(\ref{2},\ref{3}) \\
		\lbl{5} \: \phi 						\=\elim\bot(\ref{4}) \\
	\)
		\lbl{6} \: \phi 						\=\elim\lor(3-5) \\
	\end{proofbox}
	}
\end{enumerate}

\newpage
\subsection*{Question 1.2}
This subquestion concerns the use of {\it semantic} arguments in propositional
logic, i.e. argument that have to do with truth tables and valuations, in some
way.

\subsubsection*{(a) \mdseries For the following formulas, decide whether they
are {\it satisfiable} or not, and whether they are {\it valid} or not. Justify
your answer.}
\begin{enumerate}[i]
	\item
	{
	$(\neg p \imp q \land r) \land (\neg (q \land r) \imp p)$ \\\\
	We give the truth table for all valuations of the formula above. \\\\
	\begin{tabular}{ccccccc|c}
		$p$ & $q$ & $r$ &
		$q \land r$ &
		$\neg (q \land r)$ &
		$\neg p \imp q \land r$ &
		$\neg (q \land r) \imp p$ &
		$(\neg p \imp q \land r) \land (\neg (q \land r) \imp p)$ \\ \hline
		{\tt F} & {\tt F} & {\tt F} & {\tt F} & {\tt T} & {\tt T} & {\tt F} & {\tt F} \\
		{\tt F} & {\tt F} & {\tt T} & {\tt F} & {\tt T} & {\tt T} & {\tt F} & {\tt F} \\
		{\tt F} & {\tt T} & {\tt F} & {\tt F} & {\tt T} & {\tt T} & {\tt F} & {\tt F} \\
		{\tt F} & {\tt T} & {\tt T} & {\tt T} & {\tt F} & {\tt T} & {\tt T} & {\tt T} \\
		{\tt T} & {\tt F} & {\tt F} & {\tt F} & {\tt T} & {\tt F} & {\tt T} & {\tt F} \\
		{\tt T} & {\tt F} & {\tt T} & {\tt F} & {\tt T} & {\tt F} & {\tt T} & {\tt F} \\
		{\tt T} & {\tt T} & {\tt F} & {\tt F} & {\tt T} & {\tt F} & {\tt T} & {\tt F} \\
		{\tt T} & {\tt T} & {\tt T} & {\tt T} & {\tt F} & {\tt T} & {\tt T} & {\tt T} \\
	\end{tabular} \\\\
	The formula is satisfiable, since at least one valuation of the formulas
	constituent propositional atoms produce a true statement --- in this case
	we have two. By definition 1.34\cite{HR}, we have that when all the
	propositional atoms $p$, $q$ and $r$ are {\tt T}, so is the formula, and
	hence $\models (\neg p \imp q \land r) \land (\neg (q \land r) \imp p)$.
	}
	
	\item
	{
	$p \lor (p \imp p \lor r) \lor (\neg p \land (q \lor r))$ \\\\
	We give the truth table for all valuations of the formula above. \\\\
	\begin{tabular}{cccccc|c}
		$p$ & $q$ & $r$ &
		$q \lor r$ &
		$p \lor (p \imp p \lor r)$ &
		$\neg p \land (q \lor r)$ &
		$p \lor (p \imp p \lor r) \lor (\neg p \land (q \lor r))$ \\ \hline
		{\tt F} & {\tt F} & {\tt F} & {\tt F} & {\tt T} & {\tt F} & {\tt T} \\
		{\tt F} & {\tt F} & {\tt T} & {\tt T} & {\tt T} & {\tt T} & {\tt T} \\
		{\tt F} & {\tt T} & {\tt F} & {\tt T} & {\tt T} & {\tt T} & {\tt T} \\
		{\tt F} & {\tt T} & {\tt T} & {\tt T} & {\tt T} & {\tt T} & {\tt T} \\
		{\tt T} & {\tt F} & {\tt F} & {\tt F} & {\tt T} & {\tt F} & {\tt T} \\
		{\tt T} & {\tt F} & {\tt T} & {\tt T} & {\tt T} & {\tt F} & {\tt T} \\
		{\tt T} & {\tt T} & {\tt F} & {\tt T} & {\tt T} & {\tt F} & {\tt T} \\
		{\tt T} & {\tt T} & {\tt T} & {\tt T} & {\tt T} & {\tt F} & {\tt T} \\
	\end{tabular} \\\\
	The formula is satisfiable, since at least one valuation of the formulas
	constituent propositional atoms produce a true statement --- in this case
	all valuations produce a true statement. By definition 1.34\cite{HR}, it
	is trivial to see that since all valuations makes the formula true, so
	follows the semantic entailment. That is, we have that $\models p \lor
	(p \imp p \lor r) \lor (\neg p \land (q \lor r))$.
	}
\end{enumerate}

\newpage
\subsubsection*{(b) \mdseries Show that \[\models \left((\phi_1 \land \phi_2
\land \dots \land \phi_n) \imp \psi \right) \imp \left( (\phi_1 \imp \psi)
\land (\phi_2 \imp \psi) \land \dots \land (\phi_n \imp \psi) \right)\] does
{\it not} hold in general, by exhibiting a concrete counterexample formula
$\phi$ and valuation $\cal L$ showing $\phi$ is not valid.} 
Suppose that the implication does not hold, then there exists a valuation,
for which it evaluates to {\tt false}. For this to happen, we must have that
the left-hand side of the expression ($(\phi_1 \land \dots \land \phi_n) \imp
\psi$) evaluates to {\tt T}, while the right-hand side $(\phi_1 \imp \psi)
\land \dots \land (\phi_n \imp \psi)$ must evaluate to {\tt F}.

\begin{itemize}
	\item For the left-hand side, at least one of its constituents $\phi$ must be
{\tt F}. We choose $\phi_i$ to be {\tt F}.

	\item For the right-hand side, chosing $\psi$ to be {\tt F} and another constituent
$\phi_j$ to be {\tt T} makes it so.
\end{itemize}

\noindent We arrive at a formula $\phi$ that reads
\scriptsize
\begin{align*}
	\star = 
	(
		(\phi_1 \land \dots \land \phi_i \land \dots \land \phi_j \land \dots \land \phi_n) \imp \psi
	)
	\imp
	(
		(\phi_1 \imp \psi) \land \dots \land
		(\phi_i \imp \psi) \land \dots \land
		(\phi_j \imp \psi) \land \dots \land
		(\phi_n \imp \psi)
	)
\end{align*}
\normalsize
For which the truth table entries of interest reveals that it does not hold.

\begin{figure}[H]
	\center
	\begin{tabular}{ccc|c|c|c}
		$\phi_i$ & $\phi_j$ & $\psi$
		& $(\phi_1 \land \dots \land \phi_n) \imp \psi$
		& $(\phi_1 \imp \psi) \land \dots \land (\phi_n \imp \psi)$
		& $\star$ \\ \hline
		{\tt F} & {\tt T} & {\tt F} & {\tt T} & {\tt F} & {\tt F} \\
		{\tt T} & {\tt F} & {\tt F} & {\tt T} & {\tt F} & {\tt F} \\
	\end{tabular}
	\label{table:star-truth-table}
	\caption{The truth table of the $\star$ formula}
\end{figure}
