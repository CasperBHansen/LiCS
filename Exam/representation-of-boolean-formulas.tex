\section{Representation of Boolean formulas}
\subsection*{Question 6.1}
\subsection*{Question 6.2}
\subsection*{Question 6.3 \mdseries Recall that for any Boolean formula $f$,
\[ \forall x.f \eqdef f[0/x] \cdot f[1/x]
\quad \text{and} \quad
\exists x.f \eqdef f[0/x] + f[1/x] \]
Similar to these we can also define the {\it Boolean derivative} ($\partial$)
of $f$ (normally written as $\frac{\partial f}{\partial x}$, but we will here
use a shorter version) as
\[ \partial x.f \eqdef f[0/x] \oplus f[1/x] \]
Given the Boolean formula
\[ g = (x_1 x_2 x_3 + x_3 x_4) \oplus (x_2 + x_4 + x_1 x_3) \]
calculate the Boolean formula $\partial x_3.(\partial x_4.g)$. Simplify your
answer as much as possible. You must supply a line of reasoning, but you do
not need to convert into BDDs for this.
}

First, we calculate the derivative $\partial x_4.g$; \footnote{I was unable to
find any definite answer on how to interpret when no operator is in-between
two boolean variables. I'm assuming that it is an implicit $\cdot$-operation
($\land$).}
\begin{align*}
	\partial x_4.g &=
	(x_1 x_2 x_3 + x_3 (0 \oplus 1))
	\oplus
	(x_2 + (0 \oplus 1) + x_1 x_3) \\
	&=
	(x_1 x_2 x_3 + x_3 1)
	\oplus
	(x_2 + 1 + x_1 x_3) \\
	&=
	(x_1 x_2 x_3 + x_3)
	\oplus 1
\end{align*}
And then we calculate derivate of $\partial x_3.(\partial x_4.g)$, by
substituting into the equation from the last calculated derivative;
\begin{align*}
	\partial x_3.(\partial x_4.g)
	&= \partial x_3.( (x_1 x_2 x_3 + x_3) \oplus 1 ) \\
	&= (x_1 x_2 (0 \oplus 1) + (0 \oplus 1) ) \oplus 1 \\
	&= (x_1 x_2 1 + 1) \oplus 1 \\
	&= 1 \oplus 1 \\
	&= 0
\end{align*}

\subsection*{Question 6.4}